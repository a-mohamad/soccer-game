\documentclass[12pt, dvipsnames, a4paper]{article}
\usepackage{geometry}
\geometry{legalpaper, margin=0.5in}
\usepackage{xcolor}
\usepackage{xspace} 
\usepackage[normalem]{ulem}
\usepackage{vwcol}
\usepackage{cancel}
\usepackage{enumitem}
\usepackage{amsmath}
\usepackage{caption}
\usepackage{graphicx}
\usepackage{amsfonts}
\usepackage{float}
\usepackage{multicol}
\usepackage{hyperref}
\usepackage{listings}
\usepackage{textcomp}
\usepackage{lstautogobble}
\usepackage[parfill]{parskip}
\usepackage{tikz-qtree}
\usepackage{tikz}
\usepackage{hyperref}
\usetikzlibrary{decorations.pathreplacing}
\tikzset{every tree node/.style={minimum width=4cm,draw,circle},
         blank/.style={draw=none},
         edge from parent/.style=
         {draw,edge from parent path={(\tikzparentnode) -- (\tikzchildnode)}},
         level distance=1.5cm}

%% Genearl %%
\renewcommand{\thesection}{\arabic{section}}

%% For convenience %%
\newcommand{\code}[1]{\texttt{#1}}
\newcommand{\bcode}[1]{\texttt{\textbf{#1}}}
\newcommand{\balert}[1]{\textbf{\alert{#1}}}
\newcommand{\rarrow}{$\Rightarrow$}
\newcommand{\tab}[1][0.5cm]{\hspace*{#1}}
\newcommand{\deepemphasis}[1]{\underline{\textbf{\Large{#1}}}}
\newcommand{\bfemph}[1]{\textbf{\emph{#1}}}
\newcommand{\OR}[0]{\lvert \: \rvert}

%% Colours %%
\definecolor{mLightBrown}{HTML}{EB811B}
\definecolor{mLightGreen}{HTML}{14B03D}

%% Pseudocode %% 
\lstdefinelanguage{pseudo}
{
	keywords=[1]{
		class,
		new,
		loop,
		until,
		end,
		if,
		else,
		then,
		return,
		while,
		for,
		to,
		fun,
		break,
		and,
		true,
		false,
		or,
		do,
	},
	keywordstyle=[1]\color{black}\bf,
	keywords=[2] {
		invariant,
		precond,
		postcond
	},
	keywordstyle=[2]\color{blue}\bf
}

\lstset{
	language 		= 	pseudo,
	basicstyle		=	\ttfamily,
	mathescape		=	true,
	escapeinside	=	||,
	tabsize			=	2,
	numbers			=	left,
	commentstyle	=	\color{OliveGreen},
	stringstyle		=	\color{mLightBrown},
	upquote			=	true,
	morestring		=	[b]',
	moredelim		=	[l][\rmfamily\itshape]{@},
	comment			=	[l]{//},
	morecomment		=	[s]{/*}{*/},
	commentstyle=\color{Gray}\ttfamily,
	showstringspaces=	false,
	showtabs		=	false,
	autogobble
}

%% Other %%
\setcounter{secnumdepth}{5}
\setcounter{tocdepth}{5}


%**************************************************************************************************************%
%______________________________________________________________________________________________________________%

\begin{document}
\title{\textbf{EECS 3311 - Lab 05\\Design Report}}
\author{Amir Mohamad, Saniz Momin, Akif Prasla, Hasnain Saifee\\\\TA - ta name}
\date{Nov 7, 2021}
\maketitle
\tableofcontents
\clearpage

\section{Introduction}

The software project is about generating a soccer game where there are two players:
a goalkeeper and a striker. You have 60 seconds to score as many goals as possible,
after each goal is scored your score increments and the game is paused. You can resume
the game by pressing key R or by going to the menu option selecting control which has a
sub menu resume and pause. To score a goal the striker can move by pressing up down left
right key and to score a goal striker has to press space key. The goalkeeper randomly defends
the gate by moving in the left and right direction bounded by the gate's length. After the
timer becomes 0 the user gets the statistics of how many goals the striker has scored and how
many saves has the goalkeeper made. When the game is over, the user can start playing again
by navigating to the menu bar, clicking on the Game menu and clicking to a new game or exit
if the user wants to end playing. The main-goal of this software project is to use OOP concepts
and efficient software design patterns to write code efficiently and follow the DRY principle
which is (Don't repeat yourself). In this software project, to properly organize our soccer
game system, the system is divided into a set of layers where each layer focuses on a
specific aspect and has an abstraction level. MVC architectural pattern is used to divide
the system into three main subsystems where the model comprises data of the system, for
example keeping record of total goals and saves of striker and goalkeeper respectively.
The view consists of the user interface part where the data in the model subsystem can be
viewed. The controller which handles interaction between the modal and the view. The detailed
design of the software system can be seen using creational design patterns such as factory
design pattern as well as prototype design pattern which focuses on creation of objects.
For example players are created via the PlayerFactory class which has getplayer method
which takes string as parameter. All that is needed to pass is a string and it will
return the appropriate object. Doing a project with partial implementation takes time
to understand the logic and the functionality of code written. It took a while until
we all understood what is the role of each subsystem, once we got that it was easy to
implement and finish the rest. Based on our understanding of this software project we
are going to write the report with the basic idea and the designs implemented for the
top level and organization of the software project. We will create a UML diagram to make
our explanation of the design patterns used more clear. We will then explain the main
elements in this system and explain their functionalities. We will conclude our report
by giving brief overview of our journey in building this software project.

\section{System Design}
\subsection{Overview}
\begin{center}
	\begin{figure}[H]
		\hspace{50pt}
		\includegraphics[scale=0.6]{diagrams/system-diagram/system-diagram-pkg.png}
		\caption{The system design \code{SOCCERGAME}'s (example placeholder image)}
		\label{fig:systemdeisgn}
	\end{figure}
\end{center}
\clearpage

\section{Front end}
\subsection{Overview}

\subsection{Class Diagram}
\begin{center}
	\begin{figure}[H]
		\hspace{50pt}
		\includegraphics[scale=0.6]{diagrams/class-diagrams/gui-model/gui-model-cd.png}
		\caption{The class diagram of \code{SOCCERGAME}'s front end model (example placeholder image)}
		\label{fig:frontend}
	\end{figure}
\end{center}
\clearpage
\subsection{Classes}
list frontend classes here

\clearpage

\section{Back end}
\subsection{Overview}

\subsection{Class Diagram}
\begin{center}
	\begin{figure}[H]
		\hspace{30pt}
		\includegraphics[scale=.5]{diagrams/class-diagrams/shape-model/shape-model-alt-cd.png}
		\caption{The class diagram of \code{SOCCERGAME}'s player model (example placeholder image)}
		\label{fig:backend}
	\end{figure}
\end{center}
\clearpage
\subsection{Classes}
list backend classes here

\clearpage

\section{Implementation}
\subsection{Class Implementation}
disscuss class implementation
\subsection{Execution}
disscuss execution process using gradle
\section{Tools}
\subsection{Source Code}
IDE: IntelliJ IDEA 2021.2.1, to implement Java classes\\
Documentation: vscode, \LaTeX, for documentation (dessign report, etc.)
\subsection{Build}
Build: Gradle 7.2 (to manage dependencies and build)\\
Java Version: openjdk version "16.0.1" 2021-04-20\\
Unit Testing: junit-jupiter-engine:5.8.1\\
Coverage: JaCoCo library
\section{Conclusion}
conclusion


\end{document}
